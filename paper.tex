%% LyX 2.0.4 created this file.  For more info, see http://www.lyx.org/.
%% Do not edit unless you really know what you are doing.
\documentclass[12pt,english]{article}
\usepackage[T1]{fontenc}
\usepackage[utf8]{inputenc}
\usepackage{geometry}
\geometry{verbose,tmargin=1in,bmargin=1in,lmargin=1in,rmargin=1in}
\usepackage{amsmath}
\usepackage{amssymb}
\usepackage{graphicx}
\usepackage{caption}
\usepackage{subcaption}

\makeatletter

%%%%%%%%%%%%%%%%%%%%%%%%%%%%%% LyX specific LaTeX commands.
%% Because html converters don't know tabularnewline
\providecommand{\tabularnewline}{\\}

%%%%%%%%%%%%%%%%%%%%%%%%%%%%%% User specified LaTeX commands.
\usepackage{babel}
\usepackage{fancyhdr}
\usepackage{lastpage}

\makeatother

\usepackage{babel}
\begin{document}
\pagestyle{fancy} \lhead{Team \# 23813} \rhead{Page \thepage\ of
\pageref{LastPage}}

\makeatletter \let\@oddfoot\@empty \let\@evenfoot\@empty \makeatother


\title{National Pipeline System as a Solution to Future Water Shortage in the United States}


\author{Control Number: 23813}


\date{\today}
\maketitle
\begin{abstract}
\thispagestyle{empty}

In response to the potential of water shortages in the United States
by 2025, we present a series of recommended infrastructure developments
intended to most efficiently meet future demands. We first create
a projection of future water usage at the county level
based on extrapolating past data collected by the US Geological Survey,
providing a visualization of water needs across the country. Based
on these projections, and geographical data concerning the location
of each county, we determine that a water management strategy primarily
based on the construction of a national pipeline network is ideal. We model
counties as vertices and pipelines as edges in order to employ a minimum
spanning tree algorithm to determine the optimal placement of new
pipelines in order to minimize costs and redundancy. Anticipating the possibility of funding being less than that required to construct all recommended
infrastructure, we create a priority ordering of the pipelines, using
a utility function based on county population data. We also consider
the water supply of the country as a whole, providing recommendations
as to the number of de-salinization plants and reservoirs to be built to meet the change in demand.

\end{abstract}
\newpage{}

\thispagestyle{empty}

To the United States Government:

$\:$

In light of the 2012-2013 drought season, and the larger problem of growing demand for water across the nation, 
our department has conducted an
analysis of how best to satisfy the water needs of the United States. 

The first part of our analysis is a projection of future water needs. 
We used data from previous years to predict the amount of additional
water that each county will need by 2025. Creating such a projection is critical, as it informs us
of what kinds of infrastructure needs to be developed, and how significant of a problem water shortages are likely to occur in the forseeable future.

Our projections suggest that water shortages could indeed be a very significant problem leading up to 2025, in two significant ways:
On one hand, we predict that the overall water demand in the country will increase by several billion gallons per day, confined to several areas of substantial increase. We will attempt to solve this problem of total water supply by constructing additional 
de-salinization plants and reservoirs. We build these in the places that have the greatest predicted increase in their water demands.
The more significant problem however, as was the case in the drought of 2012-2013, is that even when there is enough water flowing in the United States as a whole, it might be the case that one city is flooding with water, while another is in drought.
We advise constructing a means of moving water on a large scale across the country, and this can be accomplished by constructing a network of high-volume pipelines.

However, this isn't as simple as it may seem — it isn't enough to connect only a small number of cities with pipelines, because it's possible that droughts could occur in areas of the country not connected by those pipelines.
On the other hand, the cost of building pipelines in too many locations would be prohibitive.
Our paper solves this problem by finding a way to connect all of mainland USA with the minimum possible length of pipeline.
This solution is still quite expensive, at a projected cost of \$79 billion dollars for the de-salinization plants and reservoirs, and \$92 billion dollars, for a total of \$171 billion dollars. However, given the massive cost to the economy of droughts such as the 2012-2013 drought, we contend that it will is the most cost-effective solution to the United States' water problems.

Compared to other analyses of the United States' fresh water supply, we believe that our paper provides some of the most useful results and recommendations, by focusing on the problem of fresh water being unevenly distributed across the country, rather than the problem of the United States as a whole lacking fresh water supplies.
Water management is a serious issue facing this country, and we strongly recommend that the government act in response to our findings.

For the full details of our findings, please refer to our paper,
attached below.

\newpage{}

\tableofcontents{}

\newpage{} 

\section{Introduction}

The availability of freshwater is growing increasingly severe as a
limiting factor in world development. In particular, issues of pollution,
climate change, and the possibility of increased water demand in the
future could potentially result in more frequent water shortages in
some parts of the United States.

It is projected that by 2025 the world's total demand for water will
exceed available supplies \cite{key-1}. In response to growing concerns
regarding water management, we construct a water management plan for
the United States in 2013, with the goal of developing infrastructure
to meet projected needs by 2025. We will provide recommendations as
to where new infrastructure ought to be placed in order to most efficiently
alleviate water shortages, based on a series of projections and a
graph-based model focused on constructing a pipeline network.

Our overall conclusions are as follows:
\begin{enumerate}
\item The US will not see a drastic increase in water usage. We estimate in Section \ref{total} that the overall increase in water usage by 2025 will not exceed 2 \%. In fact many regions will see a net decrease in water usage, according to our models.
\item In the absence of overall water usage growth, our water management plan focuses on being robust to short-term crises like droughts. To solve the latter we propose a national grid of high-volume pipelines to improve connectivity and transport water to areas with increased seasonal water needs.
\item Using cluster analysis we identify areas in which the growth in water demand will be most severe and recommend specific infrastructure in these regions to balance increased demand.
\item Finally, we make recommendations on how to improve sustainability and decrease environmental impact, especially pollution and the depletion of groundwater.
\end{enumerate}

\subsection{Overview of Our Approach}

We begin our paper with a projection of the likely levels of water
demand in the United States by 2025 on a county-by-county basis. Our primary source of data is the County-level water use data made publically available
by the US Geological Survey (USGS) in years 1985, 1990, 1995, 2000,
and 2005 \cite{key-2,key-3,key-4,key-5,key-6}. While making these projections, we assume that water use
equals the current water demands.

This paper attempts to address key aspects of a feasible water management
strategy, based on the available data. With our projection of expected
water needs in mind, we aim to meet demands by investing in the following
infrastructure: 
\begin{itemize}
\item De-salinization plants 
\item Water purification facilities 
\item Reservoirs and dams
\end{itemize}
However, it is not sufficient to simply have more water available
in the country's supplies as a whole. It's also critical that each
individual location be able to access that water.

 For this purpose
we aim to find the optimal locations to place pipelines.

This paper is  concerned with the construction of new water
supply infrastructure. Thus, there are some aspects of the water management
process that will not be further considered: 
\begin{itemize}
\item Public education: As with all issues of resource conservation, the
behaviour of the general public has an effect on water supplies. This
will not be considered here, for several reasons. The key issue is
that there is a lack of empirical evidence suggesting that currently
known public education strategies will work to minimize water usage
in the United States (developing such strategies further is beyond
the scope of this paper). The United States in particular has been
particularly resistant to attempts to reduce per capita water consumption\cite{key-7}.
More generally, even water education programmes considered particularly
successful, such as that of Singapore, have not managed to make a
huge impact on consumption, with an approximate 5 per cent reduction
in household water use over 6 years\cite{key-8}. Finally, 
water education programmes are likely to have a similar effects in
each location — there is no particular reason to believe they would be much more successful in any one location. This means that areas that would be most in need of
additional infrastructure taking into account such programmes would
be essentially the same as those areas most in need while not taking
into account such programmes. 
\item Maintenance costs of current infrastructure: minimizing maintenance
costs depends on technological advancement and management, but is
essentially a fixed cost irrespective of the new infrastructure we
decide to implement.
\item Cost of obtaining data: Much of our analysis depends on USGS data.
We note that the cost of obtaining this survey data is negligible
(the WaterSMART initiative, which collects much of the data, cost
\$21 million this year\cite{key-9}. In comparison, de-salinization
facilities can easily cost billions to construct\cite{key-10}) compared
to the utility of superior decision making in constructing water-related
infrastructure (among various other things the data is useful for).
\end{itemize}

\section{Modeling Infrastructure}


\subsection{Assumptions}

Due to the scale and complexity of the task of water management in
the United States, we make several simplifying assumptions in order
to obtain a workable model.
\begin{itemize}
\item Counties are thought of as discrete entities, occupying points in
space located at their centroid. This assumption enables us to think
of counties as vertices (as in a graph). This assumption can be made
without losing too much accuracy for several reasons: The United States
is split into more than 3,000 counties, with an average area of less
than 3,300 square kilometers, and average length across of less than
60 kilometers. The level of inaccuracy is sufficiently low that it
is workable (in contrast, states for instance are far too large to
be accurately modeled as points).
\item De-salinization plants, reservoirs, and purification plants are all facilities that are built at a single location, and have the intended effect of providing more water to that location. For the purposes of our modelling, we think of them as being essentially the same kind of thing.
\item When considering the costs of pipelines, only the distance of pipeline
is considered. The landscape of a region may in practice affect the
length of pipeline required, or may result in additional cost per
length of pipeline, but we shall not model this, as we could not find
data on the difficulty of pipeline construction across different types
of terrain.
\item Pipelines are thought of as bi-directional. This assumption essentially
reflects the reality of pipelines. There are a few exceptional cases
where it is significantly more difficult to send water in one direction,
but these are rare enough that they can be disregarded.
\item Pipelines have variable rates of water transmission, depending on
factors such as size. In theory it would be possible to save money on pipelines by constructing smaller pipelines in areas where water rarely needs to be transported at high rates. 
However, the amount of water that needs to be transported through any given pipe is highly variable (ie. after a sudden intense rainfall). Thus, pipelines for long-distance water transmissions are relatively stable in cost. Our estimate below for pipeline cost is a typical figure.
\item Some pipelines exist already, but we do not consider them in our model. This is partially because they are likely to have minimal impact because there are not many of them, and partially because already-existing pipelines are poorly documented, so there is no way to efficiently collect and parse such data
\end{itemize}

\subsection{Costs\label{sub:Costs}}

The prices per unit for water-related infrastructure are important
constants within our models.

We assume a constant cost per length of pipeline. Estimates for the
cost of pipeline range from approximately \$1.5 million per mile to
\$5 million per mile\cite{key-11}, with the higher figures generally
corresponding to older pipelines constructed less efficiently than
is possible today. We assume a figure of \$2 million per mile for
the purposes of our pricing estimates.

The cost of a de-salinization facility in 2013 is approximately \$14
per gallon per day\cite{key-12}.


\section{Projecting Water Demand}

To inform our water management decisions we extrapolated from the
available data from the years 1985-2005 to project the total water
usage at the county level in 2025. The available data comes in a matrix
$WU[i,j]$ with $i\in\{1985,1990,1995,2000,2005\}$ and $j$ ranging
through the US counties (indexed by FIPS codes), of total water used
per county. In the USGS data \cite{key-2,key-3,key-4,key-5,key-6}
certain peculiarities and missing data arises, for instance:
\begin{itemize}
\item Eight full states are missing water usage data in 2000 altogether.
\item Anomalous data an order of magnitude off arises commonly in the data
in year 1985.
\item Other discrepancies appear in certain counties - for example, a handful
of county FIPS codes go out of existence between 1990 and 1995, and
two new counties are introduced in between 2000 and 2005.
\end{itemize}
A second difficulty with projection is the sheer temporal distance
between the data we have, between 1985 and 2005, and the date at which
we need to model, 2025. This massive extrapolation distance means
that our projections will be highly sensitive to small changes in
model, which requires that we use sensitive outlier analysis before
applying regression analysis on the water usage data.

We apply the following approach to remove outliers from $WU[i,j]$
and project the usage, which we assume to be equal to the total demand,
on the county level in 2025. 


\subsection{Total Supply\label{total}}

We model the overall water requirements of the United States by applying
a linear regression to the total amount of water used across the country
as a whole in Table 1.

\vspace{5mm}
\begin{table}[h]\label{waterusetable}
\begin{center}
\begin{tabular}{|c|c|}

\hline 

Year & Total Water Use (Mgal/day)\tabularnewline

\hline 

\hline 

1985 & 398465\tabularnewline

\hline 

1990 & 407980\tabularnewline

\hline 

1995 & 401554\tabularnewline

\hline 

2005 & 410465\tabularnewline

\hline 

2013 (proj.) & 413695\tabularnewline

\hline 

2025 (proj.) & 419355\tabularnewline

\hline 

\end{tabular}
\caption{Total water use in 1985-2005 and projected values in 2013 and 2025.}
\end{center}
\end{table}
\vspace{5mm}

Based on the above prediction, overall water demand in the United States will increase by 5.66 billion gallons per day between 2013 and 2025. 
Multiplying by the average cost of de-salinization facilities (\$14  million per billion gallons per day) noted above, it would cost \$79.24 billion to build enough de-salinization plants to meet this demand.

\section{Meeting Local Demands}

Although the total demand is projected to increase by about $2\%$ across the United States, most of this usage increase is concentrated in a few areas. By state, the most dramatic projected increases in water usage occur in the following regions: South Florida, Arkansas and Missouri, Nebraska, Wyoming and Colorado, and California and Arizona (See Figure). Such substantial and permanent increase in water usage requires permanent increases in supply.

To solve this problem, we propose to strategically place desalination plants in the coastal regions, namely California and Florida, that require water and build reservoirs to serve inland regions, namely the Arkansas and Nebraska regions. 

\section{Optimizing Placement of Pipelines}

Our modeling of future water needs is consistent with external research
on drought locations\cite{key-13}, in that it suggests many of the
areas at risk of drought do not have the proximity to water sources
required to make the construction of de-salinization plants feasible.

The droughts in central America tend to occur in periods of minimal
rainfall, and as the surrounding rivers and groundwater aquifers also
dry up.

This suggests that there is no currently known way to locally construct
infrastructure to alleviate the water shortages in many of the areas
in which water is most needed. 

With this fact in mind, along with the knowledge that some areas in
the United States have a surplus of water, we consider a plan implemented
elsewhere in the world in a similar situation: a grid of pipelines
(aqueducts), as implemented in Queensland, Australia\cite{key-14},
in order to move water from counties with surplus water to those suffering
drought conditions.


\subsection{Modeling a Grid of Pipelines}

To optimize the placement of pipelines, we parse geographical data
for each of the counties, and produce a graph in which counties are
represented as vertices, and undirected edges are placed between adjacent
counties, with weights corresponding to the distance between the counties
being connected.

We aim to connect all of the mainland counties (excluding Alaska and
Hawaii, which our model predicts to be in little need of additional
water infrastructure) while minimizing the length of pipeline used,
and hence minimizing the cost of the project.

To do this, we implement Kruskal's minimum spanning tree algorithm
on the complete graph of counties with edges weighted by distance. 

From GPS coordinates of the (geographical) centroids of all 3109 mainland counties, we calculated the adjacency matrix $d[i,j]$ of all county pairs, weighted by geographical distance.

Running Kruskal's algorithm with a priority queue and a union-find data structure we were able to calculate the full minimal spanning tree in under a minute of running time.

\begin{figure}[h]
        \centering
        \includegraphics[width=\textwidth]{graph}
        \caption{Minimal-cost spanning tree of (continental) US counties overlaid on top of US map.}
        \label{fig:graph}
\end{figure}


The end result is a spanning tree of minimal total length connecting
all counties of the US together. This massive project would resolve
all local fluctuations in water availability, allowing high-volume
water transport between nearby counties and from distance regions
to meet seasonal needs in certain areas.

\subsection{Projected Cost of the Grid}

To estimate the cost of building a national high-volume aqueduct system
from scratch, we estimated in Section \ref{sub:Costs} using publically
available figures that the average cost per mile is approximately
\$2 million per mile as technology becomes cheaper and more efficient. 

\subsection{Priority Rank of Pipelines}

This massive pipeline project is difficult to implement within a small time period, so we made the following calculations to prioritize the building of certain pipelines over others. We prioritize based on minimizing cost and maximizing the population served by each piece of the grid.

Edges in the Minimal Spanning Tree in Figure \ref{fig:graph} are weighted inversely proportional to cost and directly proportional to the populations at each of the endpoints, and sorted. The resulting sorted table gives a full list of all $3108$ pipelines sorted by this utility metric. The final (highest priority) rows are shown in Table \ref{table:priority}.

\begin{table}[h]\label{table:priority}
\begin{center}
\begin{tabular}{|c|c|c|}
\hline
FIPS 1 & FIPS 2 & Priority \tabularnewline
\hline
48201 & 48339 & 10346.3 \tabularnewline
\hline
24005 & 24510 & 10675.4 \tabularnewline
\hline
6001  &6013 & 11019.8 \tabularnewline
\hline
34003 & 34017 & 11190.7 \tabularnewline
\hline
51153 & 51685 & 12104.5 \tabularnewline
\hline
11001 & 51013 & 12113.6 \tabularnewline
\hline
48029 & 48091 & 13787.1 \tabularnewline
\hline
34017 & 36061 & 16584.3 \tabularnewline
\hline
51059 & 51600 & 18528.8 \tabularnewline
\hline
51153 & 51683 & 22134.2 \tabularnewline
\hline
6037 & 6059  &23742.4 \tabularnewline
\hline
36047 & 36061  &27446.1 \tabularnewline
\hline
36059 & 36081  &29256.5 \tabularnewline
\hline
36005 & 36061 & 32595 \tabularnewline
\hline
36047 & 36081 & 33517.6 \tabularnewline
\hline
17031 & 17043 & 41637.4 \tabularnewline
\hline

\end{tabular}
\caption{Last 17 (highest priority) rows of the table of priority ranks of pipelines to be built.}
\end{center}
\end{table}

\section{Data Analysis and Visualization}

\subsection*{Computational Software}
All data analysis and visualization was performed using the technical computing language Julia, making use of the packages "DataFrames", "Gadfly", "Distributions", "Compose", "Graphs" all of which are available in the Julia package manager and were licensed under the MIT license. 

\subsection{Geographical Information}
\begin{figure}[h]
		\centering
        \includegraphics[width=\textwidth]{outlines}
        \caption{Visualization of Shapefile data. For details on the methods used for the Visualization see the "Visualization" section}
        \label{fig:outline}
\end{figure}

\subsubsection{Data Origin}
As the problem of water distribution and availability is very much determined by the geographic location and conditions, we obtained a dataset containing the outlines and geographic locations of every county in the United States, which we then used in our county level analysis. Figure \ref{fig:outline} shows a visualization of the contents of this data set. For the method of visualization used, see Section \ref{sec:vis}.

\subsubsection{Data Processing}
The dataset we obtained was in the proprietary, binary "Shapefile" format, that is used by many professional GIS systems. Luckily, there exists an open documentation of the format \cite{key-21}, which allowed us to implement our own parser to extract the locations, and shapes of the various counties. 

From the extracted polygons, we were able to calculate many of the desired geographical variables, such as total area of the county and it's location.

\subsection{Water Usage Information}
\subsubsection{Data Origin}

As outlined in Section 3, we made use various data points collected by the USGS (United States Geological Survey) for every single county in the United  States in the years 1985, 1990, 1995, 2000 and 2005. The data is split into 104 categories ranging from standard data points such as the total amount of water consumed in a specific specific county to more unexpected ones such as water irrigation due to golf courses. 


The data is available from the USGS website in standard ASCII tsv format.

\subsection{Visualization}
\label{sec:vis}

\begin{figure}
        \centering
        \begin{subfigure}[b]{0.9\textwidth}
                \centering
                \includegraphics[width=\textwidth]{05.png}
                \caption{2005}
                \label{fig:2005}
        \end{subfigure}
        
        \begin{subfigure}[b]{0.4\textwidth}
                \centering
                \includegraphics[width=\textwidth]{85.png}
                \caption{1985}
                \label{fig:1985}
        \end{subfigure}
        \begin{subfigure}[b]{0.4\textwidth}
                \centering
                \includegraphics[width=\textwidth]{90.png}
                \caption{1990}
                \label{fig:1990}
        \end{subfigure}
        
        \begin{subfigure}[b]{0.4\textwidth}
                \centering
                \includegraphics[width=\textwidth]{95.png}
                \caption{1995}
                \label{fig:1995}
        \end{subfigure}
       \begin{subfigure}[b]{0.4\textwidth}
                \centering
                \includegraphics[width=\textwidth]{00.png}
                \caption{2000}
                \label{fig:2000}
        \end{subfigure}
        \caption{USGS Data on total water consumption in the indicated year. The more red, the higher the water consumption of the county.}\label{fig:water}
\end{figure}

For this data, we wrote a visualization software that allows arbitrary sets of county-level data to be displayed on maps of the United States. The visualization of total water usage for every county in the years covered by the USGS data is shown in figure \ref{fig:water}


\section{The Case of Insufficient Funding}


Our model projects levels of supply and demand in 2025, and proceeds
to recommend infrastructure to meet that demand while minimizing cost.
This reflects the main precedent in major development of water supply
infrastructure in the United States, namely the \$24.6 billion Clean
Water Act \cite{key-15}, established with the aim of eliminating
all pollution of water supplies.

However, it is possible there will be insufficient funding for this
project. In this case, it will not be possible to construct all of
the recommended infrastructure. We approach this problem by creating
a priority listing, ordering the suggested developments in terms of
utility per dollar. To this end, we employ the following formula:


\section{Sensitivity and Variance}

There are two anticipated causes of significant variance in water
levels:
\begin{itemize}
\item Pathological rainfall levels (ie. extended periods of drought) 
\item Significant unexpected change in population or behavior 
\end{itemize}
Water systems in general are resistant to small variance in water
supply or demand, and our system in particular fares well in this
regard. Many of our suggested developments are particularly well equipped
to handle over-supply: our model recommends the construction of additional
pipelines and de-salinization plants — pipeline flow can be easily
reduced, while de-salinization plants can be set to operate at partial
capacity, saving electricity and avoiding over-supply.

Slight shortages of water in the United States have consistently been
dealt with successfully by implementing targeted water restrictions
\cite{key-16}. There is little argument that the limitations, largely
on gardening water, do not have a huge impact on people. 


\section{Environmental Impacts}

The core of our water management strategy is a large network of pipelines
across the country in order to shift water to locations where it is
most needed. The main reason for this is that similar approaches have
proven effective when dealing with issues of drought in inland regions
where groundwater and surface water supplies approach depletion. However,
our plan of relying mostly on pipelines also has some other benefits
over relying on other kinds of infrastructure.

While de-salinization is simply not an option at all for much of central
America, one idea we considered was the construction of additional
dams and reservoirs. One considerable environmental issue associated
with large dams is that they can cause unanticipated changes in overall
river flow\cite{key-17}, totaling potentially huge amounts in damages.

A secondary and extremely desirable effect of high national connectivity is decreased reliance on groundwater. As a case study to understand the overexploitation of aquifers, consider the High Plains aquifer, which supplies water to Colorado, Kansas, Nebraska, New Mexico, Oklahoma, South Dakota, Texas, and Wyoming, some of the most agricultural states in the US \cite{key-19}.
Due to overexploitation many areas have experienced severe water-level decline, an average of 12.8 feet. While 9 percent of areas suffered decline as severe as 50 feet, only a bare 2 percent saw any significant (over 10 feet) degree of water-level rise since  prior to development. It is estimated that 9 percent of the total storage in this massive and essential groundwater supply was exhausted in 2005.

To alleviate this chronic and unsustainable drain on our nation's precious groundwater resources, our water grid would funnel water to agricultural states from more sustainable water supplies such as desalinated seawater and surface water supplies. At very least, we can guarantee spreading the stress on aquifers much more evenly than the current wildly varying rates of exploitation.

\section{Conclusions}

Our analyses allow us to make several important conclusions and recommendations
regarding the future of water in the United States. By projecting
future increases in water, we observe that while some parts of the
United States (particularly on the coasts) are not at significant
risk of water shortage by 2025, other parts of the country, particularly
Central America, which already experienced drought in 2012-2013, are
predicted to increase their demands for water in the upcoming years,
and hence will be at risk of severe drought. Moreover, the
area of the country potentially at risk is large, and a significant
portion of the population could be affected. We conclude that there
is significant utility in investing further in water infrastructure
in the United States.

The cost of the proposed infrastructure developments are significant, at a total of $171 billion,
but must be placed in context. The cost of the 2012-2013 drought season
on the United States economy is estimated at between \$75 billion
and \$150 billion, and the potential total cost to the economy of
drought-related occurrences between 2013 and 2025 is much greater
even still. 

As a result, we firmly recommend that the entirety of our proposed
infrastructure developments, particularly the national pipeline network,
be implemented over the upcoming years.
\begin{thebibliography}{10}
\bibitem{key-1}Cai, X., Cline, S., Rosegrant, M. (2002). Global Water
Outlook to 2025. \emph{International Food Policy Research Institute.}

\bibitem{key-2}Estimated Use of Water in the United States County-Level
Data for 1985 (1955). \emph{US Geological Survey. }

\bibitem{key-3}Estimated Use of Water in the United States County-Level
Data for 1990 (1990). \emph{US Geological Survey. }

\bibitem{key-4}Estimated Use of Water in the United States County-Level
Data for 1995 (1995). \emph{US Geological Survey. }

\bibitem{key-5}Estimated Use of Water in the United States County-Level
Data for 2000 (2000). \emph{US Geological Survey.}

\bibitem{key-6}Estimated Use of Water in the United States County-Level
Data for 2005 (2005). \emph{US Geological Survey.}

\bibitem{key-7}Per Capita Water Use. Retrieved February 1, 2013,
from http://www.pepps.fsu.edu/safe/pdf/sc1.pdf.

\bibitem{key-8}Loh, J. (n.d.). Water management: Learning from
Singapore's water succcess. Retrieved February 1, 2013, from http://workingwithwater.filtsep.com/view/934/water-management-learning-from-singapore-s-water-success/.

\bibitem{key-9}2013 USGS Budget Proposal. Retrieved February 1,
2013, from http://www.usgs.gov/newsroom/article.asp?ID=3103.

\bibitem{key-10}Moran, T. (2007). Water Wars: Quenching Las Vegas'
Thirst. Retrieved February 1, 2013, from http://abcnews.go.com/Nightline/story?id=3012250.

\bibitem{key-11}Mission 2012: Clean Water. Retrieved February
4, 2013, from http://web.mit.edu/12.000/www/m2012/finalwebsite/solution/glaciers.shtml.

\bibitem{key-12}City of Carlsbad Seawater Desalination, Retrieved
February 4, 2013, from http://www.carlsbadca.gov/services/departments/water/pages/seawaterdist.aspx.

\bibitem{key-13}US Drought Monitor. Retrieved February 3, 2013,
from http://droughtmonitor.unl.edu/.

\bibitem{key-14}South East Queensland water grid. Retrieved Februrary
3, 2013, from http://www.dsdip.qld.gov.au/infrastructure-delivery/south-east-queensland-water-grid.html.

\bibitem{key-15}Burian, S., Nix, S., Pitt, R., Durrans, S. (2000).
Urban Wastewater Management in the United States: Past, Present and
Future. Retrieved February 1, 2013, from http://www.sewerhistory.org/articles/whregion/urban\_wwm\_mgmt/urban\_wwm\_mgmt.pdf.

\bibitem{key-16}Kennedy, D., Klein, R., Clark, M. (2004) Use
and Effectiveness of Municipal Water Restrictions During Drought in
Colorado. Retrieved February 1, 2013, from http://sciencepolicy.colorado.edu/admin/publication\_files/resource-296-water\_restrictions\_jawra.pdf.

\bibitem{key-17}China's Three Gorges Dam. Retrieved February
3, 2013, from https://www.mtholyoke.edu/\textasciitilde{}vanti20m/classweb/website/environmentalimpact.html.

\bibitem{key-18}http://www.wunderground.com/blog/JeffMasters/article.html?entrynum=2326

\bibitem{key-19} V.L. McGuire. Changes in Water levels and Storage in the High Plains Aquifer, Predevelopment to 2005.

\bibitem{arcgis}http://forums.arcgis.com/threads/26330-Where-can-I-find-a-shapefile-with-all-US-counties-and-FIPS-code-for-each

\bibitem{julia}http://julialang.org

\bibitem{shapefile-spec}http://www.esri.com/library/whitepapers/pdfs/shapefile.pdf
\end{thebibliography}

\end{document}
